%%%%%%%%%%%%%%%%%%%%%%%
% 		proyecto final
%%%%%%%%%%%%%%%%%%%%%%%
%-----------------------------------------------------------
% Instrucciones de compilaci�n para TexShop
% cmd + T	: latex
% cmd + ^ + B	: bibtex
% cmd + T	: latex: a�ade seccion referencias
% cmd + T	: latex: a�ade referencias cruzadas
%-----------------------------------------------------------

\documentclass[a4paper,11pt]{article}
%-----------------------------------------------------------
% Paquetes
%-----------------------------------------------------------
%\usepackage[spanish, english]{babel}
\usepackage[spanish]{babel}
%\usepackage[utf8]{inputenc}
%\usepackage[applemac]{inputenc}
\usepackage[latin1]{inputenc}
\usepackage{anysize}
\usepackage{natbib}
\usepackage{hyperref}
\usepackage{verbatim}

%-----------------------------------------------------------
% Nuevos Commandos
%-----------------------------------------------------------
% Keywords command
\providecommand{\keywords}[1]
{
  \small	
  \textbf{\textit{Palabras clave---}} #1
}

%-----------------------------------------------------------
% El Documento
%-----------------------------------------------------------
%--- M�rgenes
\marginsize{2cm}{2cm}{2cm}{2cm}

\begin{document}
%--- T�tulo
\title{Observaci�n de la interacci�n aerosol-nube}
%--- Autores
\author{Diego Bermejo-Pantale�n, Otro Autor, Otro Autor M�s}
\maketitle
%--- Resumen
\begin{abstract}
\begin{center}
%\href{https://github.com/curso-git-latex/proyecto_final.git}{\verb"https://github.com/curso-git-latex/proyecto_final.git"}
\verb"https://github.com/curso-git-latex/proyecto_final.git"
\end{center}
En este trabajo presentamos una nueva metodolog�a de observaci�n de la interacci�n aerosol -nube basada en medidas de l�dar, radar y ceil�metro incluidas en la red europea Cloudnet.
\end{abstract}
%--- Palabras Clave
\keywords{aerosol, nubes l�quidas, aci}

%--- Cuerpo
\section*{Introducci�n}
La interacci�n aerosol-nube se ha destapado como uno de los mecanismos esenciales para cuantificar el balance energ�tico en la baja atm�sfera [\cite{IPCC2014}]. Entender esta interacci�n se ha convertido en uno de los problemas a resolver dentro de la f�sica de la baja atm�sfera [\cite{Sarna2016}]. En este trabajo, presentamos una modificaci�n de la metodolog�a de \cite{Sarna2016} aplicada a la estaci�n de Granada.

\section*{Methodology}
%Dada una nube l�quida, se postula una relaci�n experimental entre la concentraci�n de gotas y la concentraci�n de aerosoles debajo de ella [\cite{}]:


\section{Results}
Here, I do not show a cite \nocite{Bermejo-Pantaleon2008a} a reference, but it appears in the. bibliography.

\section{Discussion}

\section{Conclussions}

% add all bibitems to bibliography
\nocite{*}

%--- Bibliografia
\bibliography{bibliografia}
\bibliographystyle{astron}

\end{document}