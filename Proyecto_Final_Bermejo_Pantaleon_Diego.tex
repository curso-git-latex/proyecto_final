%%%%%%%%%%%%%%%%%%%%%%%
% 		proyecto final
%%%%%%%%%%%%%%%%%%%%%%%
%-----------------------------------------------------------
% Instrucciones de compilaci�n para TexShop
% cmd + T	: latex
% cmd + ^ + B	: bibtex
% cmd + T	: latex: a�ade seccion referencias
% cmd + T	: latex: a�ade referencias cruzadas
%-----------------------------------------------------------

\documentclass[a4paper,11pt]{article}
%-----------------------------------------------------------
% Paquetes
%-----------------------------------------------------------
%\usepackage[spanish, english]{babel}
\usepackage[spanish, es-tabla]{babel}
%\usepackage[utf8]{inputenc}
%\usepackage[applemac]{inputenc}
\usepackage[latin1]{inputenc}
\usepackage{anysize}
\usepackage{natbib}
\usepackage{hyperref}
\usepackage{verbatim}
\usepackage[pdftex]{graphicx}
\usepackage{booktabs}
\usepackage{caption} 
\captionsetup[table]{skip=5pt}


%-----------------------------------------------------------
% Nuevos Commandos
%-----------------------------------------------------------
% Comando para palabras clave
\providecommand{\keywords}[1]
{
  \small	
  \textbf{\textit{Palabras clave---}} #1
}
% Entorno Figura
\newenvironment{grafico}[6]{
\begin{figure}[!htbp]
\begin{center}
\includegraphics[angle=#5, width=#6cm]
{#1} % Nombre de Archivo
\vspace{-0.3cm}
\caption[#2]{\small{#3}} % Pie de grafico
\label{#4} %Referencia
\end{center}
\end{figure}}

%-----------------------------------------------------------
% El Documento
%-----------------------------------------------------------
%--- M�rgenes
\marginsize{2cm}{2cm}{2cm}{2cm}

\begin{document}
%--- T�tulo
\title{Observaci�n de la interacci�n aerosol-nube en Granada}
%--- Autores
\author{Diego Bermejo-Pantale�n, Otro Autor, Otro Autor M�s}
\maketitle
%--- Resumen
\begin{abstract}
\begin{center}
%\href{https://github.com/curso-git-latex/proyecto_final.git}{\verb"https://github.com/curso-git-latex/proyecto_final.git"}
\verb"https://github.com/curso-git-latex/proyecto_final.git"
\end{center}
\noindent En este trabajo presentamos una nueva metodolog�a de observaci�n de la interacci�n aerosol -nube basada en medidas de l�dar, radar y ceil�metro incluidas en la red europea Cloudnet.
\end{abstract}
%--- Palabras Clave
\keywords{aerosol, nubes l�quidas, aci}

%--- Introducci�n
\section*{Introducci�n}
La interacci�n aerosol-nube se ha destapado como uno de los mecanismos esenciales para cuantificar el balance energ�tico en la baja atm�sfera \citep{IPCC2014}. Entender esta interacci�n se ha convertido en uno de los problemas a resolver dentro de la f�sica de la baja atm�sfera \citep{Sarna2016}. En este trabajo, presentamos una modificaci�n de la metodolog�a de \cite{Sarna2016} aplicada a la estaci�n Cloudnet de Granada.
%--- Estado del arte
\section*{Estado del arte}
Dada una nube l�quida, se postula una relaci�n experimental entre la concentraci�n de gotas y la concentraci�n de aerosoles debajo de ella \citep{Twomey1967} (Ecuaci�n \ref{eq:nd_na}). \cite{Feingold2003} propone un �ndice experimental para encontrar la correlaci�n entre la concentraci�n de gotas de nube y un proxy de la concentraci�n de aerosoles (Ecuaci�n \ref{eq:ACI_N}). Con medidas de radar (Figura \ref{fig:Z}) encontramos concentraci�n de gotas de nube y, con medidas de l�dar y de ceil�metro (Figura \ref{fig:beta}), encontramos un proxy para la concentraci�n de aerosoles. Con ellas, un valor para ACI puede ser estimado.

%--- Imagenes y Tablas
\section*{Im�genes y Tablas}
\begin{grafico}{imagen_Z.png}{Z}{Reflectividad medida con radar.}{fig:Z}{0}{8}\end{grafico}
\begin{grafico}{imagen_beta.png}{beta}{Retrodispersi�n atenuada medida con lidar.}{fig:beta}{0}{8}\end{grafico}
\begin{table}\begin{center}
\caption[ACI$_N$]{\small ACI$_N$.}
\label{tab:escenarios_aci}
\begin{tabular}[!]{l c}
\hline
 D�a				&      ACI$_N$  \\
 \hline
1-Enero-2020 		&	0.5     	\\
1-Febrero-2020 	&	0.8     	\\
1-Marzo-2020 		&	0.9     	\\
\ldots			& 	\ldots	\\
\hline
\end{tabular}
\end{center} \end{table}

%--- F�rmulas
\section*{F�rmulas}
\noindent Relaci�n entre concentraci�n de gotas de nubes y concentraci�n de aerosoles:
\begin{equation}
N_d = N^{\gamma}_a
\label{eq:nd_na}
\end{equation}
�ndice ACI:
\begin{equation}
ACI_N = \frac{d \, ln \, N_d}{d \, ln \, \alpha}, 0 < ACI_N < 1
\label{eq:ACI_N}
\end{equation}

%--- Bibliografia
\bibliography{bibliografia}
\bibliographystyle{astron}

\end{document}